\documentclass{article}
\usepackage{amsmath}
\usepackage[utf8]{inputenc}
\title{Innlevering 2}
\date{2016-10-30}
\author{Lowzow}

\begin{document}
	\maketitle
	\newpage

	\section{Deloppgave a}
	For å studere hvordan væskestrømmen avhenger av radius $r$, bruker vi 
	\\ $r = \sqrt{x^2 + y^2}$ \\ Først kan vi se på hvilken væskestrøm vi har i punktet
	\\ $P = (0,1) \Rightarrow P = (0,r)$

	\begin{align*}
		F_1(P) &= \Big[\frac{1}{r} \Big] \\
		F_2(P) &= \Big[1 \Big] \\
		F_3(P) &= \Big[{r} \Big] \\
		F_4(P) &= \Big[r^2 \Big] \\
	\end{align*}

	\section{Deloppgave b}
	Siden vi kan regne ut væskestrømmen utifra radiusen, vil to punkt med samme radius ha lik væskestrøm. Vi kan plassere uendelig mange punkter langs en sirkel med radius r, og alle vil ha lik væskestrøm. Dette betyr at vi ikke vil finne noen punkter som er kilder eller sluk. Dermed har hele feltet divergens lik null.
\end{document}


